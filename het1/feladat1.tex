\documentclass{article}
\usepackage[magyar]{babel}
\usepackage{t1enc}
\usepackage{graphicx}
\usepackage{xcolor}
\frenchspacing
% Komment

\begin{document}
\begin{large}

1. feladat \par
\end{large}

Együtt megoldva. \\

\begin{large}
2. feladat \par
\end{large}

\texttt{\textbf{typewriter és vastag} } \par
\textsc{\textit{kiskapitális és italic} } \par
\textsl{\textsf{slanted és sans serif} } \par

\textit{\emph{emph parancs italic szövegen belül.} } \par
\textbf{\emph{emph parancs vastag betűs szövegen belül.} } \par
\textsc{\emph{emph parancs kiskapitális szövegen belül.} } \par
Egy szó a mondaton belül, legyen \begin{large}nagybetű\end{large} míg egy másik  legyen csupa \MakeUppercase{nagybetű}.\\

\begin{large}
3. feladat \par
\end{large}
A \textbf{graphicx} csomagra.\par
\scalebox{-1}[1]{Vízszintesen tükrözött szöveg} \par
\framebox{ \scalebox{-1}[1]{Bekeretezett vízszintesen tükrözött szöveg} } \par
\scalebox{-1}[2]{Tükrözött és szétnyújtott szöveg}

\rotatebox{90}{90°-al elforgatott szöveg} \par
\rotatebox{270}{270°-al elforgatott szöveg} \par
\textcolor{white}{\colorbox{black} {invertált}} szöveg \par
\fcolorbox{red}{white}{Piros keretes szöveg}
\end{document}