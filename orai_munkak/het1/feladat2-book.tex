\documentclass[twoside]{book}
\usepackage[magyar]{babel}
\usepackage{t1enc}
\usepackage{hulipsum}
\usepackage{geometry}
\frenchspacing

% Margók (belső: 3cm, külső: 5cm, kötésmargó: 1cm)
\geometry{inner=3cm,outer=5cm}
\geometry{bindingoffset=1cm}
% Széljegyzet (3cm vastag, 0.5cm választja el a szövegtől)
\geometry{marginparwidth=3cm,marginparsep=0.5cm}

% Címsor, Tartalomjegyzék számozások
\setcounter{secnumdepth}{5} 
\setcounter{tocdepth}{5}
\renewcommand{\thefootnote}{\fnsymbol{footnote}}

\begin{document}
\title{Könyv címe}
\author{Szerző}
\maketitle

\pagenumbering{Roman}
\tableofcontents
\newpage

\pagenumbering{arabic}
\section[zagyvaság]{Első zagyvaságok}
\footnote{Ez egy lábjegyzet, a section címsorba.}
\subsection{Bla bla bla}
\hulipsum[2]
\subsection{Még több zagyvaság..}
\hulipsum[2]
\subsubsection{Zagyvaságon belüli zagyvaság}
\paragraph{Egy "paragraph"}
\subparagraph{Egy szinttel lejjebbi "subparagraph"}
\newpage
\section{Második oldalnyi zagyvaság}
\hulipsum[2-3] \linebreak
\footnote{Ez egy lábjegyzet}

{
Mi történik a számozással, ha subsubsection szintű címsort hozunk létre
közvetlenül a section szintű után? \par
Létrehozódik a section szinten belül, egy subsection pl: 0.1.-ből lesz 0.1.1.
}

\appendix
\section{Függeléken belüli rész}
\subsection{Valami szöveg...}
\subsection{Bla bla bla2}

\section{Függeléken belüli másik rész}
\subsection{Egy másik "subsection"}
\subsection{és még egy.}

\end{document}