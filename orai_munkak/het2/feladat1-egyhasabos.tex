\documentclass[12pt]{book}
\usepackage[magyar]{babel}
\usepackage[T1]{fontenc}  % Magyar karakterek és betűtípus kezelés
\usepackage{hulipsum} % Generált szöveghez
\usepackage{geometry} % Oldalméret és margók beállításához
\usepackage{fancyhdr} % Fejléc és lábléc kezelése
\usepackage{enumitem} % enumitem
\frenchspacing % Francia spacing szabályok alkalmazása

% Oldalstílus beállítása
\pagestyle{fancy}

% Fejléc és lábléc testreszabása
\fancyhead[LO,RE]{\leftmark}  % Páros oldalon a szakasz neve, páratlan oldalon az alfejezet neve
\fancyhead[RO,LE]{\thepage}   % Az oldalszám a fejléc külső sarkában
\fancyfoot[C]{Miskolci Egyetem}  % Lábléc közepére az egyetem neve
\renewcommand{\footrulewidth}{0.4pt}  % Választóvonal a lábléc fölött

% Első oldal beállítása
\thispagestyle{plain}  % Az első oldalon plain stílus, üres fejléc
\renewcommand{\footrulewidth}{0.4pt}  % A lábléc fölött legyen választóvonal

% Fejléc magasságának növelése
\setlength{\headheight}{15pt}  

% Oldalméret beállítása
\geometry{inner=3cm, outer=5cm, bindingoffset=1cm}  % Belső és külső margó, kötésmargó
\geometry{marginparwidth=3cm, marginparsep=0.5cm}  % Széljegyzet beállítása

% Címsor és tartalomjegyzék beállítások
\setcounter{secnumdepth}{5}   % Maximum szakaszszint
\setcounter{tocdepth}{5}      % Maximum szakasz szint a tartalomjegyzékben
\renewcommand{\thefootnote}{\fnsymbol{footnote}}  % Lábjegyzet szimbólumok

\begin{document}

\title{Könyv címe}
\author{Szerző}
\maketitle

% Tördelés és tartalomjegyzék
\pagenumbering{Roman}
\tableofcontents
\newpage

\pagenumbering{arabic}  % Átállás arab számokra
\chapter[zagyvaság]{Első zagyvaságok}
\footnote{Ez egy lábjegyzet, a section címsorba.}
\subsection{Bla bla bla}
\hulipsum[2]  % Generált szöveg
\subsection{Még több zagyvaság..}
\hulipsum[2]
\subsubsection{Zagyvaságon belüli zagyvaság}
\paragraph{Egy "paragraph"}
\chapter{Egy szinttel lejjebbi "subparagraph"}
\section{Második oldalnyi zagyvaság}
\hulipsum[2-3] \linebreak
\footnote{Ez egy lábjegyzet}

{
Mi történik a számozással, ha subsubsection szintű címsort hozunk létre
közvetlenül a section szintű után? \par
Létrehozódik a section szinten belül, egy subsection pl: 0.1.-ből lesz 0.1.1.
}

\appendix  % Függelék kezdete
\chapter{Függeléken belüli rész}
\subsection{Valami szöveg...}
\subsection{Bla bla bla2}

\chapter{Függeléken belüli másik rész}
\subsection{Egy másik "subsection"}
\subsection{és még egy.}

Egysoros listák:
\begin{itemize}[label=\# , before=\hspace{0pt}, after=\hspace{0pt}]
    \item Első elem
    \item Második elem
    \item Harmadik elem, és
\end{itemize}

Számozott lista:
\begin{enumerate}
    \item Első szint
    \begin{enumerate}
        \renewcommand{\labelenumi}{\alph{enumi}.}  % Kisbetűs számozás
        \item Második szint
        \begin{enumerate}
            \item Harmadik szint
        \end{enumerate}
    \end{enumerate}
\end{enumerate}

\newlist{myenum}{enumerate}{5}  % Létrehozunk egy új enum listát, ami 5-ig megy.
\setlist[myenum,1]{label=\arabic*()}   % 1. szint arab szám, zárójelben
\setlist[myenum,2]{label=\arabic*()}   % 2. szint arab szám, zárójelben
\setlist[myenum,3]{label=\roman*()}   % 3. szint római szám, zárójelben

\begin{myenum}
    \item Első elem
    \begin{myenum}
        \item Második elem
        \begin{myenum}
            \item Harmadik elem
        \end{myenum}
    \end{myenum}
\end{myenum}

\hulipsum[1]

\begin{enumerate}
    \setcounter{enumi}{2}  % Folytatjuk a számozást 3-tól
    \item Harmadik elem
    \item Negyedik elem
\end{enumerate}

\begin{enumerate}
    \item Első elem
    \item Második elem
    \item[] Ez nem lesz számozva
    \item Harmadik elem
\end{enumerate}

\end{document}