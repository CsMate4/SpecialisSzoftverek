\documentclass[12pt,aspectratio=169]{beamer}
\PassOptionsToPackage{table}{xcolor}

\usepackage{multimedia}
\usetheme{Madrid}
\usecolortheme{seahorse}

\title{Gyakorlat}
\author{Marót Máté}
\date{\today}

\begin{document}

\begin{frame}
  \titlepage
\end{frame}

\begin{frame}
  \frametitle{Gyakorlás}
  \framesubtitle{Gyakorlás}
  Ez egy bevezető dia.
\end{frame}

\begin{frame}
  \frametitle{Gyakorlás}
  Ez egy második dia szövege.
\end{frame}

\begin{frame}
  \frametitle{Gyakorlás}
  Ez egy harmadik dia szövege.
\end{frame}

\begin{frame}[fragile]
  \frametitle{Verbatim példa}
  \begin{verbatim}
  Itt egy verbatim szöveg.
  \end{verbatim}
\end{frame}

\begin{frame}[allowframebreaks]
  \frametitle{Több bekezdésnyi szöveg}
  Itt jön a hosszú zagyva szöveg, amit automatikusan széttördel több frame-re...
\end{frame}

\begin{frame}
  \frametitle{Két oszlopos dia}
  \begin{columns}[T]
    \begin{column}{0.5\textwidth}
      \begin{itemize}
        \item Első elem
        \item Második elem
      \end{itemize}
      \begin{enumerate}
        \item Első szám
        \item Második szám
      \end{enumerate}
    \end{column}
    \begin{column}{0.5\textwidth}
      \begin{figure}
        \centering
        \includegraphics[width=\textwidth]{szines.jpg} % Kép beillesztése
        \caption{Színes kép}
      \end{figure}
    \end{column}
  \end{columns}
\end{frame}

\begin{frame}
  \frametitle{Block példák}
  \begin{block}{Normál Block}
    Ez egy normál block.
  \end{block}

  \begin{alertblock}{Alert Block}
    Ez egy figyelmeztető block.
  \end{alertblock}

  \begin{exampleblock}{Example Block}
    Ez egy példa block.
  \end{exampleblock}
  
  \begin{block}{}
    Ez egy cím nélküli block.
  \end{block}
\end{frame}

\begin{frame}
  \frametitle{Tétel és bizonyítás}
  \begin{theorem}
    Ez egy tétel.
  \end{theorem}

  \begin{proof}
    Itt a bizonyítás jön.
  \end{proof}
\end{frame}

\begin{frame}[fragile]
  \frametitle{Semiverbatim példa}
  \begin{semiverbatim}
  \textbf{Példa:} \textcolor{red}{Ez egy kiemelt parancs}. 
  \end{semiverbatim}
\end{frame}

\section{Első szekció}
\subsection{Első alfejezet}

\begin{frame}
  \frametitle{Tartalomjegyzék}
  \tableofcontents[currentsection]
\end{frame}

\section{Második szekció}
\subsection{Második alfejezet}

\begin{frame}
  \frametitle{Tartalomjegyzék}
  \tableofcontents[currentsection]
\end{frame}

\begin{frame}
  \frametitle{Block overlay}
  \begin{block}{Blokk címe}
    Első szöveg \pause
    Második szöveg \pause
    Harmadik szöveg
  \end{block}
\end{frame}

\begin{frame}
  \frametitle{Áttűnések tesztelése}
  \transfade
  Tartalom az áttűnéshez...
\end{frame}

\end{document}
