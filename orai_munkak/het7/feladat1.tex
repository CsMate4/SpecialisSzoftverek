\documentclass{article}
\usepackage{ifthen}  % Feltételes műveletekhez
\usepackage{tikz}    % Ábrákhoz
\usepackage{xcolor}  % Színekhez
\usepackage{colortbl} % Színes táblázatokhoz
\usepackage{lipsum}  % Zagyva szöveghez
\usepackage{multicol} % Több oszlopos szöveghez
\usepackage{amsmath}  % Matematikai eszközök

% 1. Feladat: Saját környezet létrehozása
\newenvironment{kulcsgondolatok}[1][Kulcsgondolatok]{
    \begin{center}
    \noindent
    \textbf{\huge #1} \\[1ex]  % Cím, vastag betűkkel és középre igazítva
    \hrule % Vízszintes vonal
    \vspace{1em}  % Plusz térköz
    \begin{minipage}{0.8\textwidth}  % Doboz, szélesség 80%
    \begin{flushleft}
}{
    \end{flushleft}
    \end{minipage}
    \vspace{1em}  % Plusz térköz
    \hrule % Vízszintes vonal
    \end{center}
}

% 2. Feladat: Számláló és helyi parancs
\newcounter{gondolat}  % Számláló létrehozása

% A számláló minden section után 0-ra áll
\makeatletter
\@addtoreset{gondolat}{section}
\makeatother

% Helyi parancs létrehozása, amely kiírja és lépteti a számlálót
\newcommand{\gondolat}{
    \refstepcounter{gondolat}  % Számláló léptetése
    \noindent
    \textbf{\thegondolat}. \hspace{1em}  % Számozás
    \begin{minipage}{0.8\textwidth}
        \vspace{1ex}
        \par  % Új bekezdés kezdése
        \noindent
    \end{minipage}
}

% 3. Feladat: Táblázat váltakozó szövegszínnel
\usepackage{array}  % Az oszloptípusokhoz

% Táblázat váltakozó háttérszínek
\newcolumntype{s}{>{\columncolor[HTML]{gray!20}} p{3cm}}
\newcolumntype{f}{>{\columncolor[HTML]{white}} p{3cm}}

\begin{document}

\section{Példa}

% 1. Feladat: Saját környezet tesztelése
\begin{kulcsgondolatok}[Fontos gondolatok]
    \lipsum[1-3]  % Zagyva szöveg
\end{kulcsgondolatok}

\section{Második szekció}

% 2. Feladat: Számláló és helyi parancs használata
\begin{kulcsgondolatok}
    \gondolat Ez az első kulcsgondolat.
    \lipsum[1]
\end{kulcsgondolatok}

\begin{kulcsgondolatok}
    \gondolat Ez a második kulcsgondolat.
    \lipsum[2]
\end{kulcsgondolatok}

% 3. Feladat: Táblázat váltakozó szövegszínnel
\begin{table}[ht]
\centering
\begin{tabular}{|s|f|s|}
\hline
\rowcolor{gray!20} A & B & C \\
\rowcolor{white} D & E & F \\
\rowcolor{gray!20} G & H & I \\
\hline
\end{tabular}
\caption{Táblázat váltakozó szövegszínnel}
\end{table}

% 4. Feladat: For Each ciklusok
\section{For Each ciklusok}
\textbf{Pozitív hárommal osztható számok 50-ig:}

\foreach \x in {3,6,9,12,15,18,21,24,27,30,33,36,39,42,45} {
    \the\x \\
}

\textbf{TikZ ábra:}

\begin{tikzpicture}
\foreach \x/\y in {0/0, 2/2, 4/4} {
    \draw (\x,\y) circle (1);
}
\end{tikzpicture}

% 5. Feladat: Cikk-cakk
\section{Cikk-cakk}
\begin{multicols}{2}

\newcounter{num}
\setcounter{num}{1}
\whiledo{\value{num} < 61}{
    \the\num \ifthenelse{\value{num} mod 3 = 0}{ cikk}{}
    \ifthenelse{\value{num} mod 5 = 0}{ cakk}{}
    \par
    \stepcounter{num}
}

\end{multicols}

\end{document}